% Options for packages loaded elsewhere
\PassOptionsToPackage{unicode}{hyperref}
\PassOptionsToPackage{hyphens}{url}
%
\documentclass[
  11pt,
  %ignorenonframetext,
  aspectratio=169]{beamer}
\input{glyphtounicode}
\pdfgentounicode=1

\usepackage{umk}
\usepackage{relsize}
\usepackage{pgfpages}
\beamertemplatenavigationsymbolsempty
% Prevent slide breaks in the middle of a paragraph
\widowpenalties 1 10000
\raggedbottom
\setbeamertemplate{part page}{
  \centering
  \begin{beamercolorbox}[sep=16pt,center]{part title}
    \usebeamerfont{part title}\insertpart\par
  \end{beamercolorbox}
}
\setbeamertemplate{section page}{
  \centering
  \begin{beamercolorbox}[sep=12pt,center]{part title}
    \usebeamerfont{section title}\insertsection\par
  \end{beamercolorbox}
}
\setbeamertemplate{subsection page}{
  \centering
  \begin{beamercolorbox}[sep=8pt,center]{part title}
    \usebeamerfont{subsection title}\insertsubsection\par
  \end{beamercolorbox}
}
\AtBeginPart{
  \frame{\partpage}
}
\AtBeginSection{
  \ifbibliography
  \else
    \frame{\sectionpage}
  \fi
}
\AtBeginSubsection{
  \frame{\subsectionpage}
}
\usepackage{amsmath,amssymb}
\usepackage{iftex}
\ifPDFTeX
  \usepackage[T1]{fontenc}
  \usepackage[utf8]{inputenc}
  \usepackage{textcomp} % provide euro and other symbols
\else % if luatex or xetex
  \usepackage{unicode-math} % this also loads fontspec
  \defaultfontfeatures{Scale=MatchLowercase}
  \defaultfontfeatures[\rmfamily]{Ligatures=TeX,Scale=1}
\fi
\usepackage{lmodern}
\ifPDFTeX\else
  % xetex/luatex font selection
\fi
% Use upquote if available, for straight quotes in verbatim environments
\IfFileExists{upquote.sty}{\usepackage{upquote}}{}
\IfFileExists{microtype.sty}{% use microtype if available
  \usepackage[]{microtype}
  \UseMicrotypeSet[protrusion]{basicmath} % disable protrusion for tt fonts
}{}
\makeatletter
\@ifundefined{KOMAClassName}{% if non-KOMA class
  \IfFileExists{parskip.sty}{%
    \usepackage{parskip}
  }{% else
    \setlength{\parindent}{0pt}
    \setlength{\parskip}{6pt plus 2pt minus 1pt}}
}{% if KOMA class
  \KOMAoptions{parskip=half}}
\makeatother
\usepackage{xcolor}
\newif\ifbibliography
\setlength{\emergencystretch}{3em} % prevent overfull lines
\providecommand{\tightlist}{%
  \setlength{\itemsep}{0pt}\setlength{\parskip}{0pt}}
\setcounter{secnumdepth}{-\maxdimen} % remove section numbering
\ifLuaTeX
\usepackage[bidi=basic]{babel}
\else
\usepackage[bidi=default]{babel}
\fi
\babelprovide[main,import]{polish}
% get rid of language-specific shorthands (see #6817):
\let\LanguageShortHands\languageshorthands
\def\languageshorthands#1{}
\usepackage{qrcode}
\ifLuaTeX
  \usepackage{selnolig}  % disable illegal ligatures
\fi
\usepackage{bookmark}
\IfFileExists{xurl.sty}{\usepackage{xurl}}{} % add URL line breaks if available
\urlstyle{tt}
\hypersetup{
  pdftitle={AI, kod i Ty -- badanie dla programistów},
  pdflang={pl-PL},
  hidelinks,
  pdfcreator={LaTeX via pandoc}}

\begin{document}

\begin{frame}%[noframenumbering,plain]
  \centering
  %\includegraphics[width=\textwidth,keepaspectratio]%
  \includegraphics[width=\paperwidth,height=\paperheight]%
  {AI_kod_i_ty}
\end{frame}


\begin{frame}
  \frametitle{AI, kod i Ty -- badanie dla programistów}
  %\begin{columns}
  %\begin{column}{1.0\textwidth}
  Na czym polega badanie
  \begin{itemize}
  \item W trakcie dwóch sesji (każda po 90 minut):
    \begin{itemize}
    \item
      będziesz rozwiązywać zadanie programistyczne\\
      oparte o~realistyczny projekt w~Python
    \item
      skorzystasz z~AI albo z~dokumentacji
    \item
      monitorujemy Twój proces za pomocą \emph{eye trackera}
    \end{itemize}
  \item Co zyskujesz?
    \begin{itemize}
    \item 150 zł za każdą sesję (łącznie 300 zł)
    \item nowa wiedza o~swoim stylu pracy
    \item udział w~badaniu nad przyszłością programowania
    \end{itemize}
  \end{itemize}
  %\end{column}
  %\end{columns}
\end{frame}

\begin{frame}
  \frametitle{AI, kod i Ty -- uczestniczenie w badaniu}
  \vspace*{4mm}
  \begin{columns}
  \begin{column}{0.72\textwidth}
  \begin{itemize}
  \item Gdzie i kiedy?
    \smallskip
    \begin{itemize}
    \item
      Wydział Matematyki i~Informatyki UMK w~Toruniu\\
      ul.~Chopina 12/18, 87--100 Toruń, \textbf{sala F004}
    \item
      Terminy ustalanie indywidualnie\\
      (do końca listopada 2025 r.)
    \end{itemize}
  \end{itemize}
  %\begin{columns}
  %  \begin{column}{0.72\textwidth}
  \bigskip
  \begin{itemize}
  \item Jak się zgłosić?
    \smallskip
    \begin{itemize}
    \item
      Ankieta rekrutacyjna:\\
      \url{https://forms.gle/hapB9a3hJYFLgdbf7}
    \item
      e-mail od \texttt{ncusi@mat.umk.pl}\\
      z dalszymi informacjami
    \end{itemize}
  \end{itemize}
  \end{column}
  \begin{column}{0.28\textwidth}
    \qrcode[height=100px]{https://forms.gle/hapB9a3hJYFLgdbf7}
  \end{column}
  \end{columns}

  % \vspace*{1cm}
  \vfill
  {%\relsize{-1}
    \centering
    
    \url{https://ncusi.github.io/eye-tracking-recruitment/}

  }
\end{frame}

\end{document}
